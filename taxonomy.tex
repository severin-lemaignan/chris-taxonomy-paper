\documentclass{article}

\usepackage[utf8]{inputenc}
\usepackage[english]{babel}
\usepackage{hyperref}
\usepackage{graphicx}
\usepackage{paralist} % for inline lists

\usepackage[draft]{fixme} % to add side notes to other editors

\newcommand{\ie}{{\textit{i.e.~}}}
\newcommand{\cf}{{\textit{cf~}}}
\newcommand{\eg}{{\textit{e.g.~}}}

\title{Building blocks for cooperation: A conceptual framework for human-robot interaction systems}

\author{Felix Warneken et al.}

\begin{document}

\maketitle
\tableofcontents

\begin{abstract}

Cooperation is at the core of human social life. Humans have developed
cognitive and behavioural skills to interact with each other in various types of
cooperative activities ranging from simple acts of helping each other to large
scale cooperative practices that involve division of labor among multiple
agents. One major challenge for research on human-robot interaction is to build
artificial agents that can successfully participate in this important area of
human social life by, for instance, assisting humans in everyday tasks or
collaborating safely and flexibly with human agents in a common workspace. Here
we present an overview and analysis of different types of cooperative behaviours
that very young children are able to engage. This is motivated by the idea that
understanding the basic cooperative behaviours and underlying cognitive skills
of human children enables us to not only identify the building blocks of the
cooperative behaviours for human development, but also provide us with guidance
for the kinds of skills that robots should possess to cooperate with humans.
Therefore, the current paper has three objectives: First, we provide a
taxonomy of the basic cooperative behaviours in human children,
distinguishing between prosocial behaviours (helping, sharing, informing)
and collaboration (especially problem-solving which includes division of
labor). Second, we provide hypotheses about the underlying cognitive
representations that humans use to perform these behaviours, namely an
ability to represent individual psychological states for prosocial
behaviours and the ability to represent joint intentions for acts of
collaboration. This is supposed to provide some guidance for the
implementation of these skills in robots, both concerning the cognitive
architecture and empirical tests of their skill level. Third, we provide
illustrative examples of robotic platforms that master some of these
cooperative behaviours to help assess the state of the art in the field of
human robot interaction.

\end{abstract}

%%%%%%%%%%%%%%%%%%%%%%%%%%%%%%%%%%%%%%%%%%%%%%%%%%%%%%%%%%%%%%%%%%%%%%%%%%%%%%%%
%%%%%%%%%%%%%%%%%%%%%%%%%%%%%%%%%%%%%%%%%%%%%%%%%%%%%%%%%%%%%%%%%%%%%%%%%%%%%%%%
\section{Introduction}


\paragraph{Rationale} Understanding the fundamental cooperative behaviours that young
children engage in will inform robotics striving to build artificial systems
for human-robot cooperation. The building blocks of human cooperative behaviours
observed in children are a good model for the building blocks for robotic
systems that engage in cooperative interactions with humans.

\paragraph{Human cooperative behaviours} First, our paper provides a
descriptive and conceptual analysis of human cooperative behaviours.

We start by providing descriptions of the basic cooperative behaviours that young
children engage in, namely \emph{helping}, \emph{informing}, \emph{comforting},
\emph{sharing}, and \emph{collaborating}. These are the first steps that humans
take towards becoming the highly cooperative species-members that we see in
adult humans.

Our conceptual analysis is accompanied by examples from empirical studies with
children, comprising the experimental scenarios as well as the behavioural
measures used to determine the skills that children bring to bear.

Although we suggest that the types of cooperative behaviours that we
describe are distinct, we do not claim that these behaviours always occur in
isolation. As a matter of fact, due to the complexity of social life,
agents might perform several of these cooperative behaviours during a
single episode. Specifically, agents might only succeed if they perform one
cooperative behaviour as part of a different cooperative behaviour (\eg help
the other agent as part of a collaborative activity). It could even be
argued that one reason why humans (including human children) are so
proficient at cooperating is exactly because they are able to shift between
different types of behaviours. However, the fact that these behaviours can
and sometimes have to co-occur certainly doesn't mean that they cannot be
differentiated conceptually and might rely on distinct cognitive processes.

Second, we provide suggestions for the cognitive architecture needed to perform
the different cooperative behaviours. While the main goal in this paper is a
descriptive one, we allude to hypotheses about the cognitive capacities that
underlie cooperative behaviours in humans as well. 

With the description of the different types of cooperative behaviours we intend
to facilitate research to test hypotheses about the underlying cognitive
structure in humans\fxnote{facilitate research on humans?}. We hope that these
broad strokes might be helpful for roboticists in thinking about ways of
implementing these behaviours in robots.

It should be noted that we do not propose that there could be a
single or best way of implementing these capacities in robots. Depending on
whether the engineer's goal is re-create human-like cognition as an
avenue to better understand humans or to create an artificial agent that
just shows analogous behaviours originating in very different processes, our
descriptions might provide normative guidance or just more or less helpful
heuristics.

We also provide a typology that can be regarded as being located at a
medium level of abstraction. 

Our typology is more general than concrete scenarios or tasks
such as moving around a chair to enter through a door or moving around
a pillar to move down a narrow hallway) and more specific than global
challenges (such as moving through space or giving objects)

Ideally, this will facilitate the design of different platforms
that are comparable despite the specifics of the implementation and the
concrete environment and objects involved, allowing for a meaningful
comparison across different robotic platforms with regard to their
capacity to perform successfully in relevant cooperative interactions
(Dautenhahn, 2010, p. 700).

In short, this might help in achieving the goal to develop
artificial systems that display skills which are generalizable across
different specific situations.

\paragraph{Human-robot interaction examples} We provide illustrative examples
of state-of-the-art robotic agents that have some of the components necessary
to socially interact with humans in one or several of the cooperative
interactions described in the taxonomy

This, we hope, will inform the reader about possible ways of implementing these
capacities in robots and obviously also identify gaps in the area of
human-robot interaction and thus function as a guideline for future research:
Where are we in the field and where should we go next?


%%%%%%%%%%%%%%%%%%%%%%%%%%%%%%%%%%%%%%%%%%%%%%%%%%%%%%%%%%%%%%%%%%%%%%%%%%%%%%%%
%%%%%%%%%%%%%%%%%%%%%%%%%%%%%%%%%%%%%%%%%%%%%%%%%%%%%%%%%%%%%%%%%%%%%%%%%%%%%%%%
\section{A descriptive and conceptual analysis of cooperative behaviours}

We use the term \emph{cooperation} to refer to social behaviours that are -- at
least in part -- aimed at facilitating another agent's goal achievement, in
contrast to purely individual behaviours that are aimed only at fulfilling the
agent's own individual goal.Examples for the latter would be actions such as
obtaining an object for oneself, changing ones own emotional state, or
competing with others\footnote{It should be noted that similar terms have been
used in biology to differentiate social behaviours along the lines of fitness costs
and benefits that are conveyed to actors and recipients of social behaviours. By
contrast, our definition focuses on the influence of concrete actions and
psychological states (\ie benefits in terms of whether behaviours allow or
hinder agents to achieve their goals) and we remain agnostic about the impact
of these behaviours on evolutionary fitness.}. Cooperation comprises
\emph{prosocial behaviours} that are aimed at acting on behalf of another
agent's individual goal and \emph{collaborative behaviours} in which
multiple agents jointly interact to achieve a joint goal. We will first define
and provide examples for prosocial behaviours, followed by a more thorough
discussion of collaborative behaviours as a form of cooperation that is distinct
from purely prosocial behaviours.

%%%%%%%%%%%%%%%%%%%%%%%%%%%%%%%%%%%%%%%%%%%%%%%%%%%%%%%%%%%%%%%%%%%%%%%%%%%%%%%%
\subsection{Prosocial behaviours}

\begin{figure}
\center
\includegraphics[width=0.9\columnwidth]{figs/prosocial_behaviours.png}
\caption{Prosocial behaviours. Thick black arrows denote prototypical type of
intervention for respective type of problem. Thin grey arrows denote
alternative intervention strategies.}
\label{fig-prosocial}
\end{figure}

\subsubsection{Definitions}

One important distinction is between problem recognition and type of
intervention. In order to act prosocially, the donor has to be able to identify
the problem that the recipient is facing. Once the problem is identified, the
donor must then decide whether to act on the other's behalf, and what
strategy to use in order to intervene appropriately. The distinction between
problem recognition and intervention is essential because either of these
processes might account for the failure to respond prosocially. Specifically,
an agent might fail to act prosocially because it did not realize that there
was a problem in the first place, or because it identified the problem, but is
unwilling or unable to act accordingly. An agent might not be able to read the
emotional cues which signal pain or distress in the recipient, and thus see no
reason to intervene. Or the agent might be very well aware of the other's
distress, but not know how to alleviate it. Thus, even if for the sake of the
argument assume that a person is generally motivated to act prosocially,
prosocial behaviour can only occur if the donor possesses the cognitive and
behavioural capacities to both recognize the problem and have the necessary
skills to intervene.

This is illustrated in Figure~\ref{fig-prosocial}, in
which at time 1, a recipient's intended goal (what the person wants to be the
case) and the actual state of the world do not align. A potential prosocial
donor has to be able to identify this problem of a goal-world mismatch, and
then choose an appropriate type of intervention. We distinguish four types of
problems that agents might encounter: An \emph{obstacle} to the successful completion
of an instrumental act (such as an object is out-of-reach, an obstacle
obstructs a path, something does not open, etc.), a \emph{lack of information} (an
object is misplaced, one cannot remember a person's name, etc.), a \emph{negative
emotion} (someone is hurt or distressed about an event), or a \emph{lack of resource}
(\eg food, money, or a tool to complete a task).

Once the type of problem is
identified, the donor can choose to act prosocially in various ways. For
example, the most straightforward way for the donor to intervene on behalf of
another person's failed action-goal is to take an action itself (such as
handing an out-of-reach object or pull harder at the cork screw when a friend
failed to open a wine bottle). Similarly, when another person fails to find
the desired object, the most salient type of intervention is to inform the
person about the actual location of the object. This is denotated in our
diagram with thick black arrows. However, donors might have different types of
intervention at their disposal to intervene with various types of goals. For
example, when the person cannot find its pen, the donor might lend it its own
rather than informing about the fact that its pen is in the other room.
This is represented by the various thin grey arrows in
Figure~\ref{fig-prosocial}.

Ultimately, a successful intervention should
contribute to a reduction in the gap between the recipient's goal and the
actual state of the world: the out-of-reach object is within reach, the person
can now see the location of the misplaced object, she feels less sad or is less
poor.

Taken together, we identify four prototypical prosocial behaviours with roots in
early human development: 

\begin{itemize}

\item In \emph{instrumental helping} situations, an agent performs a concrete
act that contributes to the recipient's fulfillment of the recipient's
individual action-goal. What elicits the helping behaviour and serves as the
criterion for the successful completion of the helping act is a conative state
of the recipient: the unfulfilled goal which, through the helper's
intervention, is fulfilled at the end of the interaction. Examples are holding a
door open for someone whose hands are full (helping her complete the goal of
entering a room) or giving an out-of-reach object the recipient is
unsuccessfully reaching for (helping her obtain an otherwise unobtainable
object).

\item \emph{Informing} refers to situations in which the problem for the
recipient is due to a lack of information that would allow it to complete an
action, satisfy its need, alleviate its distress, etc. What elicits informing
is the recognition on the side of the cooperator that the recipient lacks a
certain piece of information. Typical situations might be the inability to find
a misplaced object, not finding food during foraging, or desperately looking
for mom after hurting ones finger. Thus, in these cases an actor can intervene
by providing the missing piece of information, by \eg pointing to the
location of the object the person is looking for or telling about the
location.

\item In \textit{comforting} situations (often also called \emph{empathic
intervention} or \emph{emotional helping}), rather than helping with a concrete
action-goal, the purpose is to change the recipients (negative) emotional
state. Thus, what elicits the helping act is the negative emotional state of
the recipient, and the act is terminated once the negative emotional state is
successfully altered, resulting in a neutral or positive emotional state.
Examples are hugging someone who is in distress (to comfort the other) or
fixing a broken object that was the cause of the distress.

\item \textit{Sharing} differs from helping in that rather than performing an
action that facilitates the recipient's goal-completion, the actor provides
concrete resources. Specifically, the problem for the recipient consists in the
lack of a certain resource (such as food) and the actor intervenes by giving up
(part of) a resource that it has in its possession. Concerning the
elicitation of a sharing act, one can differentiate between \emph{active
sharing}, which involves cases in which agent A actively transfers the resource
from the actor to the recipient, and \emph{passive sharing} that characterizes
situations in which the recipient is taking the object that the actor has in
its possession, with the actor noticing, but tolerating this transfer. 

\end{itemize}


\subsubsection{Empirical studies with young children}

\fxnote{The main question here is how much detail we should provide
here. We could focus on the main results (``children help by picking up
objects from 14 months of age'') or provide a fair amount of detail about the
setups and measures (perhaps including photos or schematic drawings). You guys
should tell us what you think is most desirable for roboticists reading this
paper. Maybe they want a short review of the different tasks that exist or
it's enough to have very short summaries and they can then look up the
original papers?}

\paragraph{Instrumental helping}

From 14 months on, infants provide help for others, \eg, open doors for
another person when her hands are occupied, or fetch objects which are out of a
needy person's reach (\eg, Warneken \& Tomasello, 2006)

With about three years of age, children provide help to a peer more often
in a collaboration that outside of it.  This will be addressed in the following
(2.2 Collaborative behaviours) in more detail

\paragraph{Comforting}

18- and 24-month-old children display concern and emotional distress when
another person is harmed (\eg, when someone had just destroyed her artwork) and
help or share with this person more than with a person who had not been harmed
(\eg Vaish, Carpenter \& Tomasello, 2009) 

\paragraph{Informing}

Infants at around one year of age use pointing gestures \emph{informative}ly,
\ie in order to indicate to another person the location of the object the
person is looking for (\eg, Liszkowski, Carpenter, Striano, \& Tomasello, 2006).

\paragraph{Sharing}

Infants sometimes give toys to their parents (\eg, Hay, 1979) and peers
(Hay et al., 1991), presumably mainly to share interest and fun. 

However, usually sharing in those early situations is not costly, \ie
infants actually do not have to give up the toy for good. Costly sharing is
more difficult -- when given a windfall of items and asked to share, even 3- and
4-year-olds often keep the majority for themselves (even though they do give
away some; Lane \& Coon, 1972; Fehr, Bernhard, \& Rockenbach, 2008; Moore,
2009; Rochat et al., 2009). 

One exception is the context of collaboration that encourages equal
sharing already among 3-year-olds (Warneken, Lohse, Melis, \& Tomasello, 2010;
Hamann, Warneken, Greenberg, \& Tomasello, 2011). For example, when pairs of
3-year-old children work together on a task by pulling in ropes jointly in
order to gain toys, they make sure that both of them receive an equal share in
the end -- and they do so more often after collaboration than in parallel work
situations, where they still sit next to each other but pull only their own
individual rope, or in windfall situations, where they receive rewards without
contributing any work. 

\subsubsection{Prosocial behaviours: Conclusion}

These prosocial behaviours are aimed at another person's individual goal,
emotion or epistemic state\fxnote{Maybe briefly define 'epistemic state'}. Even though one could say that for the helping act,
the helper adopts the other person's goal of retrieving an object or removing
an obstacle to let the other person enter, or the helper resonates with the
negative emotional state of the other person when empathizing, the critical
piece is the other person's individual goal or the other person's emotional
state that lead to the actor's intervention in the first place and is also
the criterion against which to decide whether the intervention was successfully
completed\fxnote{Should split this sentence!}. For example, if the recipient, for some reason, now decides that it no
longer wants the dropped object because it found a different one, it is also no
longer necessary for the actor to try to retrieve it.

In other terms, even if
the goals match in a given situation, the actor's goal depends on the
recipient's individual goal, rather than existing independently. Similarly,
in cases of emotional helping, what should determine whether comforting
behaviours are still appropriate is the emotional state of the recipient and not
the emotional state of the intervening actor.

This has potentially important implications for the decision about the
cognitive architecture that needs to be implemented in robotic systems.
Specifically, according to this model, the representation of individual
psychological states (\ie individual goals, emotions, or epistemic states) is
necessary and sufficient to successfully engage in these prosocial behaviours
concerning the social-cognitive capacities required\fxnote{\[concerning...required\]: does 
this sentence qualifies 'prosocial behaviours'? I do not understand very well.}. 

On the other hand, the representation of individual psychological states is not
sufficient to engage in sophisticated collaborative activities as are typical
in human social interaction. Specifically, a different class of cooperative
behaviours are collaborative activities in which two or more agents perform
actions together to bring about a joint goal. This is particularly important to
differentiate from cases of instrumental helping, in which the helper
identifies the other person's individual goal, whereas in cases of
collaboration, two agents co-represent a joint goal and execute their
respective actions in order to achieve it. 

%%%%%%%%%%%%%%%%%%%%%%%%%%%%%%%%%%%%%%%%%%%%%%%%%%%%%%%%%%%%%%%%%%%%%%%%%%%%%%%%
\subsection{Collaborative behaviours}

\subsubsection{Definitions}

An important form of cooperative behaviour in humans are cases in which
individuals pool their efforts to produce outcomes that neither individual
would be able to achieve alone. This is apparent in large scale collaborative
practices such as building a skyscraper, which depends upon division of labor
among a group of individuals, but also for simpler tasks such as lifting a
large sofa. In either case, the task is achieved as the common effect of
several individuals. One can in principle distinguish between cases in which
pooling efforts is necessary (no single individual is strong enough to lift the
dining table) from cases in which it is more efficient (carrying a mattress
alone is cumbersome). For the current purposes, however, we are concerned with
describing interactions that can be considered collaborative as based upon the
way in which agents interact, whether collaboration is necessary or
facilitative.

One major question is \emph{What enables individuals to engage in these collaborative
activities}. Importantly, research shows that many human collaborative
activities are not simply the summative effect of individual actions, but that
individual actions are interrelated in specific ways, creating a distinct from
of human cooperation. We will use the term \emph{joint collaborative activity}
to denote this type of social interaction. Roughly speaking, in joint
collaborative activities, individuals perform their individual action-plans
(intentions) in pursuit of a joint goal. For example, it is my goal to wash the
lettuce and your goal to make a dressing, but both actions are executed in
pursuit of the goal of preparing a salad. It has been argued that the ability
to engage in joint collaborative activities is an essential if not
species-unique feature of human social interaction which is based upon a
specific form of cognitive representation: \emph{joint intentions} (Tomasello et al.,
2005). The collaborative activities that we observe in humans are often based
upon these joint intentions, which are markedly different from the
representation of individual intentions underlying other types of social
interaction that are more widespread in the animal kingdom. Roughly speaking,
individual and joint intentions\footnote{Different authors use different terms
to speak of basically the same concept: joint, shared, collective intentions or
we-intentions. We prefer joint because it highlights the interlocking of
intentions (\ie action-plans) and because in other discussions of human
cooperation, the term \textit{shared intention} has not been clearly differentiated
from \textit{sharing} in the sense of resource sharing. Namely, having shared
intentions is not akin to having intentions to share.\par } can be
differentiated as follows:

{\bf Individual intention}:
{\it I intend to bring about goal x by means of y.} or {\it You intend to bring about goal x by means of y.}

{\bf Joint intention}:
{\it We intend to bring about goal x by means of me doing y$_1$ and you
doing y$_2$.}

In the behavioural sciences, it is often difficult to determine whether an
activity that is performed by multiple agents should be conceived of as a joint
collaborative activity that is based upon joint intentions or merely as a the
common outcome of individual intentions. For example, each individual agent
might be acting on an individual intention towards an individual goal, and even
though the outcome emerges from the combined efforts of the agents, they are
not necessarily acting jointly. In other words, what qualifies as a plural
activity (Butterfill, 2011) does not necessarily qualify as a joint
collaborative activity\fxnote{Give example?}. These issues have created
debate in psychology, cognitive science and among philosophers, trying to best
define the conceptual criteria for joint collaborative activity and devising
empirical methods to capture the underlying cognitive processes (Bratman,
Gilbert, Tollefsen, Pacherie, Searle, Tomasello{\dots}). In addition, it has to
be elaborated how these sophisticated underlying cognitive processes are
applicable to young children. Even at an early age, children show very
competent social interactions with others, which appear on its face
intentional, joint collaborative action long before these skills can be
empirically demonstrated (for a review see Brownell 2011). 

Based in particular upon the work by Bratman (1992, 2009) and
Tomasello (Tomasello et al., 2005), we speak of collaboration proper when
(roughly) the following criteria are met:

\begin{enumerate}

\item{\it Mutual responsiveness}: We are mutually responsive to each other
in the sense that my actions are in part determined by your actions (and vice
versa).

\item{\it Joint goal}: We both have a joint goal and thus represent our
respective individual actions as part of an overarching goal.

\item{\it Commitment to joint goal}: We mutually coordinate our plans of
actions (our intentions) in pursuit of this goal. This entails the mutual
commitment to perform 
\begin{inparaenum}
\item ones individual actions in a way that does not
interfere with the other's actions and \item provide help when the other agent
encounters problems that prevent her from performing her actions.
\end{inparaenum}

\end{enumerate}

These criteria provide guidance for behavioural tests to determine whether
certain social behaviours that humans (and other animals) engage in can be
characterized as joint collaborative activity. Moreover, it is a tool to
describe how the capacity to engage in collaboration emerges over development.
In recent years, several experimental studies have been conducted to
investigate the development of collaboration in young children. We will provide
a summary of these studies in order to show which basic abilities are present
early in life, and to give illustrative examples of the kind of experimental
situations that have been created to empirically test which behavioural skills
and cognitive representations they utilize to collaborate.

We will focus on
studies in which agents engage in instrumental acts, such as collaborative
problem-solving or interactive games because they represent an important part
of early collaboration and seem most relevant for the kinds of actions that are
relevant for human-robot interaction\footnote{Thus, in parallel to the
different forms of prosocial behaviours aimed at influencing individual states
described above (instrumental helping, sharing, informing), one might find
equivalent forms in collaboration in all three domains (such as collaborative
resource sharing without an instrumental component or can describe the exchange
of information as a collaborative act).}. 

\subsubsection{Empirical studies with children}

\fxnote{Same question here: how much detail should we provide?}

\paragraph{Mutual responsiveness and behavioural coordination}

With a very skillful and competent partner like an adult, young children
(1-year-olds) acquire their first experience on tightly coordinated actions in
time and space being executed by two different bodies and directed towards
external objects (Bakeman and Adamson 1984; Hay 1979; Eckerman et al. 1989;
Warneken and Tomasello 2007)

Especially interesting are novel situations in which children cannot rely
on practices routines (scripts). Such novel tasks were developed by Warneken et
al. (2006), where two individuals had to coordinate predefined roles such that
the activity could be executed together. The tasks required a partner and
correct spatio-temporal coordination. In one task, a mechanical cylinder had to
be pushed upwards to make a toy accessible to the partner. In another task, a
cube was thrown into one of two available tubes and had to be caught by the
partner, operating also on that selected tube. 18-month-olds interacted rather
uncoordinated with the adult while 24-month-olds were already quite coordinated
and also anticipated the partners actions.

Starting at 30 months of age, pairs of same-aged children (peers) are
able to anticipate the partner and to act in accordance to the others spectrum
of possible actions in a novel cooperative problem-solving task (Brownell 1990,
1991). 

A cognitively more challenging task is to be able to plan the
coordination of roles in anticipation of the action execution. Therefore,
Steinwender et al. (under review) tested whether children are capable of
dividing up the labor ahead of a task, demonstrating that they are able to
represent \emph{both} roles in their interrelation and within one
representational format. This study indicates that 3-year-olds are able to plan
ahead a collaborative task only under certain conditions, whereas 5-year-olds
proficiently divide up the roles. In their task, peers had to correctly select
two different tools in order to subsequently jointly manipulate a
problem-solving apparatus. 3-year-olds were only successfully selecting the
correct tools when the partner had already selected his tool and thus
determined the choice ha had to make. But when the first selecting individual
had to predict what the partner would do before he could make his won
individual plan, 3-year-olds were unsuccessful.

\paragraph{Representation of joint intentions and commitment to joint goal}

Warneken et al. (2006) investigated whether children will re-engage their
partner (joint goal) when he quits the interaction or continue playing
individually (no joint goal). Already at 18 months children had formed a joint
goal and communicated in order to bring to partner back into the game. These
results are supported by the result that 21- and 27-month old children
differentiating between a partner who is unwilling or unable to continue,
focusing not on the behavioural outcome alone, but taking into account the
intentions underlying the behaviour (Warneken, Graefenhain, Tomasello 2011).
Thus, in a collaborative activity, children appear to remind the partner of his
contribution to the joint activity, and take into account the partner's
intention to participate.

The question whether children are committed to the joint collaboration
and will thus support their partner when he needs help or cannot complete his
role, had been investigated by taking into account children's ability to help
the partner completing his role or even taking over the other's role. The
rationale behind is that the overall goal is to do something together and thus
the roles of both partner will be represented within one representational
format (Tomasello et al. 2005). When it is known how the roles interrelate and
influence each other, then it should be no problem at all to help the partner
with his role, take over the partner's role and even to plan how the labor is
going to be divided between the partners.

Hamann et al. (2011) tested whether children themselves are committed to
the joint goal (rather than reminding the partner of the partner's commitment
to the joint goal). In this study, peers faced the situation that during a
necessary collaborative interaction towards two individual rewards, one reward
was accessible for one child before the other could complete her role and
receive her reward. Children working with a shared goal in mind, where each
role is part of one overarching goal, will continue their part until the
partner had achieved her share. At 3.5 years of age, children were found to
have established a joint goal and continued the interaction, while
2.5-year-olds did not support their partner until they received their reward.

\subsubsection{Summary}

Collaborative behaviours as defined by Bratman (1992) and Tomasello et al.
(2005) require sophisticated representational skills that go beyond the
representation of individual goals and execution of individual action-plans.
Recent advances in developmental psychology provide novel experimental
paradigms to assess to what extent children have the capacities to engage in
collaborative interactions and whether they are based upon joint intentions. 

%%%%%%%%%%%%%%%%%%%%%%%%%%%%%%%%%%%%%%%%%%%%%%%%%%%%%%%%%%%%%%%%%%%%%%%%%%%%%%%%
%%%%%%%%%%%%%%%%%%%%%%%%%%%%%%%%%%%%%%%%%%%%%%%%%%%%%%%%%%%%%%%%%%%%%%%%%%%%%%%%
\section{Implementations in robots}

{\bfseries\itshape We would first re-iterate why we think that these
developmental studies can provide some guidance for appropriate scenarios,
behavioural measures, and perhaps cognitive architecture in robotics.}

{\bfseries\itshape This is where you guys come in: what examples should we
include here? We think that we have to restrict it to illustrative examples and
cannot provide a comprehensive overview of ALL the robots and all the task that
exist in the field of robotics.}

%%%%%%%%%%%%%%%%%%%%%%%%%%%%%%%%%%%%%%%%%%%%%%%%%%%%%%%%%%%%%%%%%%%%%%%%%%%%%%%%
\subsection{Current robotic platforms}

One way we can consider organizing this section would be first to describe
between 5-10 existing systems, including the CHRIS I and II systems, and
Dominey \& Warneken (in press!), the work from LAAS on shared planning and other
relavant systems .In order to keep the scope focused, we can consider systems
that explicitly take into account the self-other distinction.

%%%%%%%%%%%%%%%%%%%%%%%%%%%%%%%%%%%%%%%%%%%%%%%%%%%%%%%%%%%%%%%%%%%%%%%%%%%%%%%%
\subsection{Implemented cooperation behaviours in current architectures}

we could then try to systematically analyse whether and how each of the
enumerated prosocial and collaborative behaviors are addressed in these
systems.  This would be similar to section 2.2.2 but with robotics studies.

The result of this analysis will likely indicate that we have a long way to go,
particularly in allowing the robot the perceptual capacities to know the
human's goals, emotional states etc.

%%%%%%%%%%%%%%%%%%%%%%%%%%%%%%%%%%%%%%%%%%%%%%%%%%%%%%%%%%%%%%%%%%%%%%%%%%%%%%%%
\subsection{Open research issues}

then we can probably come up with a set of key research issues in order to make
significant advances in this area.

%%%%%%%%%%%%%%%%%%%%%%%%%%%%%%%%%%%%%%%%%%%%%%%%%%%%%%%%%%%%%%%%%%%%%%%%%%%%%%%%
%%%%%%%%%%%%%%%%%%%%%%%%%%%%%%%%%%%%%%%%%%%%%%%%%%%%%%%%%%%%%%%%%%%%%%%%%%%%%%%%
\section{Glossary}
%TODO: turn it into a real Latex glossary

\paragraph{Goal-directed action}

\textbf{Goal}: We refer to goal as a (mental) representation of a desired state
of the environment. Usually, this is a change in the environment, \ie the
desired outcome of an action. But the goal can also be to preserve a certain
state of the environment. In our terminology, `goal' is not the same as a
goal-object. 

\textbf{Intention}: An intention is the plan of action the agent chooses and
commits herself to in pursuit of a goal.  The plan of action thus encompasses
both the goal and the means to achieve it. The intention can be a purely
individual intention, in which the agent only represents his own means to
achieve an individual goal or the individual actions can also be part of a
joint intention (see below). 

\textbf{Outcome}: The outcome or result of an action refers to the actual and
observable environmental state post action. Diagnostic situations are those in
which the goal (the internally represented and intended outcome) and the actual
outcome (the observable state of the environment) do not match.

\begin{itemize}

\item \textbf{Success} = The action changed the state of the environment so that
outcome and goal now correspond.  This usually leads to feelings of
satisfaction by the agent.

\item \textbf{Failed attempt} = The action did not result in a change of state of the
environment. That is, outcome and goal do not match due to the lack of change
regarding the critical component of the action. This is usually accompanied by
expressions of disappointment.

\item \textbf{Accident} = The action resulted in a change of the environment, but
caused an unintended outcome. That is, outcome and goal do not match because
the action lead to a wrong outcome. The agent usually responds with surprise.

\end{itemize}

For details see Tomasello et al. 2005.\textit{(We could add a figure similar to
that by Tomasello)}

\paragraph{Structural aspects of collaborative activities}

\paragraph{Parallel vs. complementary actions}

\textit{Parallel actions}= Both agents perform the same kind of action.

Example 1: We hold the two ends of a heavy board and lift it together.

Example 2: We pull two parallel ropes to move a heavy objects towards us.

Example 3: We both play on our drums.

\textit{Complementary} actions = Both agents perform different but interrelated
actions.

Example 1: I hold the frame steady so that you can nail it to the wall.

Example 2: I hold and turn the table top so that you can attach the four legs
one by one.


%%%%%%%%%%%%%%%%%%%%%%%%%%%%%%%%%%%%%%%%%%%%%%%%%%%%%%%%%%%%%%%%%%%%%%%%%%%%%%%%
%%%%%%%%%%%%%%%%%%%%%%%%%%%%%%%%%%%%%%%%%%%%%%%%%%%%%%%%%%%%%%%%%%%%%%%%%%%%%%%%
\section{Role reversal and ``bird's eye view''}

In a collaborative activity with two complementary actions, both agents can
perform either action and reverse them if necessary. For example: I can hold
the frame and you use the hammer to nail it to the wall or you hold the frame
and I use the hammer. The idea is that if two agents are able to engage in
role-reversal, it shows that they not only have a representation of their own
action, but also of the other person's action as related to ones own action.
Thus, both agents have an agent-independent representation of the collaborative
activity with two social roles and either agent can adopt either role. In other
words, they look down at the social interaction from a bird's eye perspective.

Steinwender et al (in preparation) used complementary problem-solving tasks and
reported that the ability to flexibly switch roles is already present at 18
months but also that the amount of role reversal increase significantly to 24-
and  to 36-months of age.

\end{document}
